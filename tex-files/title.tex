% Page heads
\markboth{XX et al.}{Building Blocks of Large Scale Stream Processing: the Past, Current, and Future}

% Title portion
\title{Building Blocks of Large Scale Stream Processing: the Past, Current, and Future}
\author{
XX
\affil{UIUC}
YY
\affil{UIUC}
}


\begin{abstract}

Various Stream Processing engines have been developed, both from research and industry communities, as a near real-time paradigm of processing massive amount of data in parallel.  These systems are tailored specifically for a wide range of use-cases from latency sensitive (\Fix{ex ?}) to high throughput demanding (\Fix{ex?}) and .  
Most of the building blocks used among systems are common, but designed with different guarantees and goals in mind. 

In this article we present a survey of the main building blocks commonly employed in current most wide-spread Stream Processing engines. This article aims to provide a better insight both for system developers, when making design decisions, and application developers when deciding among various configuration options.
This article outlines the essential building blocks in Stream Processing engines, and signifies the importance of each, and demonstrates the interaction among them.
 
 In a catalog style we survey each building block by discussing  the current and past techniques used for each building block and the trade-offs in each technique. We describe the design of \Fix{5} most popular Stream Processing engines, and the building blocks used in each. Finally, we discuss the future road-map of stream processing, the remaining challenges and the missing building blocks with efficient techniques.


\end{abstract}


%
% The code below should be generated by the tool at
% http://dl.acm.org/ccs.cfm
% Please copy and paste the code instead of the example below. 
%

\begin{CCSXML}
	<ccs2012>
	<concept>
	<concept_id>10010520.10010553.10010562</concept_id>
	<concept_desc>Computer systems organization~Embedded systems</concept_desc>
	<concept_significance>500</concept_significance>
	</concept>
	<concept>
	<concept_id>10010520.10010575.10010755</concept_id>
	<concept_desc>Computer systems organization~Redundancy</concept_desc>
	<concept_significance>300</concept_significance>
	</concept>
	<concept>
	<concept_id>10010520.10010553.10010554</concept_id>
	<concept_desc>Computer systems organization~Robotics</concept_desc>
	<concept_significance>100</concept_significance>
	</concept>
	<concept>
	<concept_id>10003033.10003083.10003095</concept_id>
	<concept_desc>Networks~Network reliability</concept_desc>
	<concept_significance>100</concept_significance>
	</concept>
	</ccs2012>  
\end{CCSXML}

\ccsdesc[500]{Computer systems organization~Embedded systems}
\ccsdesc[300]{Computer systems organization~Redundancy}
\ccsdesc{Computer systems organization~Robotics}
\ccsdesc[100]{Networks~Network reliability}

%
% End generated code
%

% We no longer use \terms command
%\terms{Design, Algorithms, Performance}

\keywords{\Fix{todo}}

%\acmformat{Gang Zhou, Yafeng Wu, Ting Yan, Tian He, Chengdu Huang, John A. Stankovic,
%	and Tarek F. Abdelzaher, 2010. A multifrequency MAC specially
%	designed for  wireless sensor network applications.}
% At a minimum you need to supply the author names, year and a title.
% IMPORTANT:
% Full first names whenever they are known, surname last, followed by a period.
% In the case of two authors, 'and' is placed between them.
% In the case of three or more authors, the serial comma is used, that is, all author names
% except the last one but including the penultimate author's name are followed by a comma,
% and then 'and' is placed before the final author's name.
% If only first and middle initials are known, then each initial
% is followed by a period and they are separated by a space.
% The remaining information (journal title, volume, article number, date, etc.) is 'auto-generated'.

\begin{bottomstuff}
%	This work is supported by the National Science Foundation, under
%	grant CNS-0435060, grant CCR-0325197 and grant EN-CS-0329609.
%	
%	Author's addresses: G. Zhou, Computer Science Department,
%	College of William and Mary; Y. Wu  {and} J. A. Stankovic,
%	Computer Science Department, University of Virginia; T. Yan,
%	Eaton Innovation Center; T. He, Computer Science Department,
%	University of Minnesota; C. Huang, Google; T. F. Abdelzaher,
%	(Current address) NASA Ames Research Center, Moffett Field, California 94035.
\end{bottomstuff}

\maketitle